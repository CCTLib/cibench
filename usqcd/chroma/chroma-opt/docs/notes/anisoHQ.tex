\documentclass[12pt]{article}
\usepackage{amsmath}

% Somewhat wider and taller page than in art12.sty
\topmargin -0.4in  \headsep 0.0in  \textheight 9.0in
\oddsidemargin 0.25in  \evensidemargin 0.25in  \textwidth 6.5in

\footnotesep 14pt
\floatsep 28pt plus 2pt minus 4pt      % Nominal is double what is in art12.sty
\textfloatsep 40pt plus 2pt minus 4pt
\intextsep 28pt plus 4pt minus 4pt

\begin{document}

\newcommand{\half}{\frac{1}{2}}
%\newcommand{\be}{\begin{equation}}
%\newcommand{\ee}{\end{equation}}
\newcommand{\be}{\begin{displaymath}}
\newcommand{\ee}{\end{displaymath}}
\newcommand{\bea}{\begin{eqnarray}}
\newcommand{\eea}{\end{eqnarray}}
\newcommand{\bdm}{\begin{displaymath}}
\newcommand{\edm}{\end{displaymath}}
\newcommand{\<}{\langle}
\renewcommand{\>}{\rangle}
\newcommand{\Tr}{\mbox{Tr}}

\centerline{\bf \Large Heavy Quarks on Anisotropic Lattices}
\vskip 5mm

These notes are based, and sometimes refer to Tim Klassen's draft paper
of the same title.

\section{Conventions}

Tim considers actions of the form $\sum \bar \psi Q \psi$, on anisotropic
lattices with $a_k=a$ for $k=1,2,3$ and (renormalized) anisotropy
$\xi = a / a_0$, with
\bea
Q &=& m_0 + \nu_0 \nabla_0 \gamma_0 - \half r_0 a_0 \Delta_0
 + \sum_k \left( \nu \nabla_k \gamma_k- \half r a \Delta_k \right) \nonumber \\
&& - \frac{a}{2} \Big[ \omega_0 \sum_k \sigma_{0k} F_{0k}
 + \omega \sum_{k<l} \sigma_{kl} F_{kl} \Big ] ~,
\label{eq:Q_generic}
\eea
where
\bea
\nabla_\mu f(x) &=& \frac{1}{2a_\mu} \bigg[ U_\mu(x) f(x+\mu) -
 U^\dagger_\mu(x-\mu) f(x-\mu) \bigg ] \\
\Delta_\mu f(x) &=& \frac{1}{a_\mu^2} \bigg[ U_\mu(x) f(x+\mu) +
 U^\dagger_\mu(x-\mu) f(x-\mu) - 2f(x) \bigg ] .
\eea
Tim also introduces the ``Wilson operator''
\bea
W_\mu \equiv \nabla_\mu - \frac{a_\mu}{2} \gamma_\mu \Delta_\mu ~.
\eea

Actually, Tim restricts the analysis always either to the choices
$\nu_0 = 1 = r_0$ ($\nu$-tuning) or $\nu = 1 = r$ ($\nu_0$-tuning).
To make use of the projection property of $(1 \pm \gamma_\mu)/2$
one needs $\nu_0 = r_0$ and $\nu = r$. Then $Q$ becomes
\bea
Q = m_0 + \nu_0 W_0 \gamma_0 + \nu \sum_k W_k \gamma_k
 - \frac{a}{2} \bigg[ \omega_0 \sum_k \sigma_{0k} F_{0k}
 + \omega \sum_{k<l} \sigma_{kl} F_{kl} \bigg ] ~.
\label{eq:Q_W}
\eea
In terms of dimensionless variables $\hat \psi = a^{3/2} \psi$,
$\hat W_\mu = a_\mu W_\mu$ and $\hat F_{\mu\nu} = a_\mu a_\nu F_{\mu\nu}$
\bea
a_0 Q = a_0 m_0 + \nu_0 \hat W_0 \gamma_0 +
 \frac{\nu}{\xi_0} \sum_k \hat W_k \gamma_k -
 \half \bigg[ \omega_0 \sum_k \sigma_{0k} \hat F_{0k} +
 \frac{\omega}{\xi_0} \sum_{k<l} \sigma_{kl} \hat F_{kl} \bigg ] ~.
\label{eq:Q_Tim}
\eea
The choice of the factors $1/\xi_0$, with $\xi_0$ the bare anisotropy,
rather than $1/\xi$ is a convention.

In {\tt SZIN} conventions we use, instead of $a_0 Q$, the operator
\bea
{\cal M} = 1 - \kappa \bigg[ {\overline W}_0 + \frac{\rho_s}{\xi_0}
 \sum_k {\overline W}_k + c_T \sum_k \sigma_{0k} \hat F_{0k} +
 \frac{c_R}{\xi_0} \sum_{k<l} \sigma_{kl} \hat F_{kl} \bigg ] ~,
\label{eq:Q_SZIN}
\eea
where
\bea
{\overline W}_\mu = (1-\gamma_\mu) U_\mu(x) \delta_{x+\mu,y}
 + (1+\gamma_\mu) U^\dagger_\mu(x-\mu) \delta_{x-\mu,y} ~.
\eea
The fermions have been rescaled as $\hat \psi = \sqrt{2 \lambda} \tilde \psi$,
so that $a_0 Q = {\cal M} / (2 \lambda)$.
We note that
\bea
{\overline W}_\mu = 2 - 2 \hat W_\mu \gamma_\mu
\eea
and find
\bea
\lambda &=& \frac{1}{2 [a_0 m_0 + \nu_0 + 3 \nu / \xi_0 ]} ~, \nonumber \\
\kappa &=& \lambda \nu_0 = \frac{\nu_0}{2 [a_0 m_0 + \nu_0 + 3 \nu / \xi_0 ]}
 = \frac{1}{2 [a_0 m_0 / \nu_0 + 1 + 3 \nu / (\nu_0 \xi_0) ]}~,  \\
\rho_s &=& \frac{\nu}{\nu_0} ~, \qquad
 c_T = \frac{\omega_0}{\nu_0} ~, \qquad c_R = \frac{\omega}{\nu_0} ~. \nonumber
\label{eq:convert}
\eea

In the {\tt SZIN} code, the parameter $\rho_s$ is denoted as {\tt Wilsr\_s},
and $c_T$ and $c_R$ as {\tt ClCoeffT} and {\tt ClCoeffR}, respectively.


\section{Tuning of the classical action}

At the classical level there is no distinction between bare and
renormalized anisotropy and the renormalized anisotropy $\xi$ will be
used. The tuning will be done for the dispersion relation in a weak
background field.

The results of the tuning will be given only for $\nu_0 = r_0$, {\it i.e.},
with the projection property in the time direction. They will not be given
as function of the bare mass $m_0$ but rather the exact mass (or {\em pole}
mass) $m = E(0) = \log(1+a_0 m_0/\nu_0) / a_0$.

At the classical level --- or when done fully non-perturbative --- the two
tuning methods, ``$\nu$-tuning'' and ``$\nu_0$-tuning'', are of course
equivalent, being simply related by a rescaling of the fermion fields.

\subsection{$\nu$-tuning}

Here one keeps $\nu_0 = 1 = r_0$ fixed. Without background field, the lattice
dispersion relation becomes the continuum dispersion relation, up to terms
of order ${\cal O}(a^2 p^4)$ for (see the appendix and Klassen (3.9))
\bea
\label{eq:nu_tune1}
\nu^2(a_0 m, \xi, r) &=& {\rm e}^{a_0 m} \frac{\sinh(a_0 m)}{a_0 m}
 - r \xi \sinh(a_0 m) \nonumber \\
r(a_0 m, \xi, \nu) &=& \frac{1}{\xi} \Biggl( \frac{{\rm e}^{a_0 m}}{a_0 m}
 - \frac{\nu^2}{\sinh(a_0 m)} \Biggr) \\
\nu(a_0 m, \xi, \rho \equiv \frac{r}{\nu}) &=&
 \sqrt{ {\rm e}^{a_0 m} \frac{\sinh(a_0 m)}{a_0 m} +
 \biggl( \frac{\rho}{2} \xi \sinh(a_0 m) \biggr)^2 }
 - \frac{\rho}{2} \xi \sinh(a_0 m) ~. \nonumber
\eea
The last form, with $\rho=1$, is the relevant one for actions with the
full projection property such as (\ref{eq:Q_W}), (\ref{eq:Q_Tim}) or
(\ref{eq:Q_SZIN}).

To get the continuum dispersion relation in a constant background field
at zero momentum, up to terms of order ${\cal O}(a^2 F^2)$, requires
the choices (see Klassen (3.12))
\bea
\omega_0(a_0 m, \xi, \nu) &=& \frac{1}{\xi} \Biggl(
 \frac{{\rm e}^{a_0 m}}{a_0 m} - \frac{\nu}{\sinh(a_0 m)} \Biggr) \nonumber \\
\omega(a_0 m, \xi, \nu) &=& \frac{1}{\xi} \Biggl(
 \frac{{\rm e}^{a_0 m}}{a_0 m} - \frac{\nu^2}{\sinh(a_0 m)} \Biggr) ~.
\label{eq:nu_tune2}
\eea
Note that $\omega(a_0 m, \xi, \nu) = r(a_0 m, \xi, \nu)$!

The mass dependence of some of the coefficients for the anisotropic clover
action, {\it i.e.,} with the full projection property, are shown in Klassen's
Figs. 3 and 4.

\subsection{$\nu_0$-tuning}

Here we keep $\nu = 1 = r$ fixed, and, to simplify the analysis, we also
take $\nu_0 = r_0$, {\it i.e.,} we enforce the full projection property.
Then, from the dispersion relation without background field we obtain
\bea
\frac{1}{\nu_0(a_0 m, \xi)} =
 \sqrt{ {\rm e}^{a_0 m} \frac{\sinh(a_0 m)}{a_0 m} +
 \biggl( \frac{\xi}{2} \sinh(a_0 m) \biggr)^2 } - \frac{\xi}{2} \sinh(a_0 m) ~.
\label{eq:nu0_tune1}
\eea
$\nu_0$ is simply the inverse of $\nu$ from $\nu$-tuning for the case of
full projection property, $\nu=r$.

{}From the dispersion relation in a background field, but at vanishing
momentum, we get
\bea
\frac{\omega_0(a_0 m, \xi)}{\nu_0} &=& \frac{1}{\xi} \Biggl(
 \frac{{\rm e}^{a_0 m}}{a_0 m} - \frac{1}{\nu_0\sinh(a_0 m)} \Biggr)
 \nonumber \\
\frac{\omega(a_0 m, \xi)}{\nu_0} &=& \frac{1}{\xi} \Biggl(
 \frac{{\rm e}^{a_0 m}}{a_0 m} - \frac{1}{\nu^2_0\sinh(a_0 m)} \Biggr)
 \equiv \frac{1}{\nu_0} ~.
\label{eq:nu0_tune2}
\eea
In particular, we find $\omega(a_0 m, \xi) = 1$, independent of $a_0 m$
and $\xi$! A plot of $\omega_0(a_0 m, \xi)$ is shown as Fig. 5 in Klassen's
draft, though he labeled it as $\omega_0(a_0 m, \xi) / \nu$. But $\nu$ has
been set to 1 here.


\section{Tuning beyond the classical level}

Here we are interested in anisotropic lattices with a fixed renormalized
anisotropy $\xi$. We assume that the bare anisotropy $\xi_0$ has already
been determined from the Euclidean invariance of the static potential.
We note that to quite good accuracy $\xi = \frac{u_0}{u} \xi_0$, with
$u_0$ and $u$ the average temporal and spatial link in Landau gauge
(with $\xi_0$ used in the gauge condition), respectively.

We start by tadpole improving, considering the action with full projection
property, eq.~(\ref{eq:Q_Tim}), only. We note that $\tilde W_\mu =
\hat W_\mu \gamma_\mu - 1$ is linear in the gauge field $U_\mu$, while
$F_{\mu\nu}$ contains $U^2_\mu U^2_\nu$. Hence, tadpoling we write
eq.~(\ref{eq:Q_Tim}) as
\bea
a_0 Q &=& \left( a_0 m_0 + \nu_0 + \frac{3\nu}{\xi_0} \right) +
 \nu_0 u_0 \left( \frac{\tilde W_0}{u_0} \right) +
 \frac{\nu u}{\xi_0} \sum_k \left( \frac{\tilde W_k}{u} \right) \nonumber \\
&& - \half \left[ \omega_0 u^2_0 u^2 \sum_k \sigma_{0k} \left(
 \frac{\hat F_{0k}}{u^2_0 u^2} \right) + \frac{\omega u^4}{\xi_0}
 \sum_{k<l} \sigma_{kl} \left( \frac{\hat F_{kl}}{u^4} \right) \right] ~.
\eea
Factoring out a $u_0$ and substituting $\xi$ for $\xi_0 u_0/u$ we get
\bea
\frac{1}{u_0} a_0 Q &=& \left( \frac{a_0 m_0}{u_0} + \frac{\nu_0}{u_0} +
 \frac{3\nu}{\xi u} \right) + \nu_0 \left( \frac{\tilde W_0}{u_0} \right) +
 \frac{\nu}{\xi} \sum_k \left( \frac{\tilde W_k}{u} \right) \nonumber \\
&& - \half \left[ \omega_0 u_0 u^2 \sum_k \sigma_{0k} \left(
 \frac{\hat F_{0k}}{u^2_0 u^2} \right) + \frac{\omega u^3}{\xi}
 \sum_{k<l} \sigma_{kl} \left( \frac{\hat F_{kl}}{u^4} \right) \right] ~.
\label{eq:Q_Tim_tp}
\eea
Now, tadpole improving consists in interpreting the combined coefficients
in front of the respective terms as the classical coefficients, {\it i.e.},
$\omega^{tp}_0 u_0 u^2 = \omega^{cl}_0$ and $\omega^{tp} u^3 = \omega^{cl}$,
while the $u$-factors in the operators are ignored, or set to 1.
Thus we set
\bea
\omega^{tp}_0 = \frac{\omega^{cl}_0}{u_0 u^2} ~, \qquad
\omega^{tp} = \frac{\omega^{cl}}{u^3} ~.
\label{eq:w_tp}
\eea
We note that $\nu$ and $\nu_0$ do not get any tadpole factors. But the
bare mass changes
\bea
a_0 m^{tp}_0 = u_0 a_0 m^{cl}_0 + \nu_0 (u_0-1) + \frac{3\nu}{\xi}
 \frac{u_0}{u} (u-1) ~.
\eea

At the classical level, $\nu$ or $\nu_0$ were functions of the anisotropy
and the quark pole mass. In QCD, we determine them non-perturbatively from
the dispersion relation of the pion. We define an ``effective velocity
of light'' $c({\bf p})$ by
\bea
c({\bf p})^2 = \frac{E({\bf p})^2 - E(0)^2}{{\bf p}^2}
 = \xi^2 \frac{a_0^2 E({\bf p})^2 -a_0^2 E(0)^2}{a^2 {\bf p}^2} ~,
\eea
where in the last expression all quantities are dimensionless. For the
tuning we want $c(0)=1$. To obtain $c(0)$, we extrapolate $c({\bf p})$,
assuming $c({\bf p}) - c(0) = {\cal O}(a^2 {\bf p}^2)$. With the other
parameters kept fixed, we determine $\nu$ or $\nu_0$ from the condition
that $c(0)=1$, within errors.

What do we take for the classical values of $\omega$ and $\omega_0$ which
we then tadpole improve? I would impose the classical relations between
$\omega$, $\omega_0$ and $\nu$ or $\nu_0$.

For $\nu$-tuning ($\nu_0 = 1 = r_0$, $r = \nu$) this gives $\omega^{cl} = \nu$,
while $\omega^{cl}_0$ needs to be obtained from eq.~(\ref{eq:nu_tune1}) and
(\ref{eq:nu_tune2}). From (\ref{eq:nu_tune1}), using $\rho=1$ we can get
the pole mass $a_0 m$ for which the measured $\nu$ would be the classical
value and then use this in (\ref{eq:nu_tune2}) to get $\omega^{cl}_0$.

Klassen used, instead, $\omega^{TK} = 1$ and $\omega^{TK}_0 = \half
\left( 1 + \frac{1}{\xi} \right)$, the value at $a_0 m = 0$, before
tadpoling. We note that depending on the value of $\xi$ and the determined
value of $\nu$, Klassen's choices can be smaller are bigger than our
``classical'' ones. This might explain the large $\xi$-dependence that
Klassen obtained in his Fig. 10. A comparsion of Klassen's choices and our
``classical'' ones is given in Table~\ref{tab:nu_tune}. We have used the
$\nu$-values found by Klassen and given in his Table 3.

\begin{table} \centering
\begin{tabular}{| l | l | l || l | l || l | l |}
\hline
$\xi$  & $\beta$ & ~$\nu$ & $\omega^{TK}$ & $\omega^{TK}_0$ & $\omega^{cl}$ &
 $\omega^{cl}_0$ \\ \hline
 3  &   5.5  & 0.830 & 1.0 & 0.666 & 0.830 & 0.6296 \\
    &   5.6  & 0.810 & 1.0 & 0.666 & 0.810 & 0.6260 \\
    &   5.7  & 0.800 & 1.0 & 0.666 & 0.800 & 0.6245 \\
    &   5.8  & 0.790 & 1.0 & 0.666 & 0.790 & 0.6231 \\
\hline
 2  &   5.5  & 1.163 & 1.0 & 0.75  & 1.163 & 1.2083 \\
    &   5.6  & 1.060 & 1.0 & 0.75  & 1.060 & 1.0794 \\
    &   5.7  & 0.990 & 1.0 & 0.75  & 0.990 & 0.9864 \\
    &   5.8  & 0.925 & 1.0 & 0.75  & 0.925 & 0.8882 \\
\hline
% 1  &   5.54 & 2.439 & 1.0 & 1.0   & 2.439 & 3.3043 \\
 1  &   5.7  & 1.600 & 1.0 & 1.0   & 1.600 & 2.0927 \\
\hline
\end{tabular}
\caption{Comparison of clover coefficients, before tadpole improving,
from $\nu$-tuning. Note that $\omega^{TK} = 1$ for all cases. The $\nu$-values
are form Klassen's Table 3.}
\label{tab:nu_tune}
\end{table}

For $\nu_0$-tuning ($\nu = 1 = r$, $r_0 = \nu_0$) this gives $\omega^{cl} = 1$,
as used by Klassen, while $\omega^{cl}_0$ needs to be obtained from
eq.~(\ref{eq:nu0_tune1}) and (\ref{eq:nu0_tune2}), similarly to the previous
case. Here, Klassen again used $\omega^{TK}_0 = \half \left( 1 + \frac{1}{\xi}
\right)$. A comparsion of Klassen's choices and our ``classical'' ones
is given in Table~\ref{tab:nu0_tune}. We have used the $\nu_0$-values found
by Klassen and given in his Table 4. We see that Klassens choice is always
smaller than our ``classical'' choice, which can be seen also from Klassen's
Fig. 5.

\begin{table} \centering
\begin{tabular}{| l | l | l || l || l || l |}
\hline
$\xi$  & $\beta$ & ~$\nu_0$ & ``$1/\nu$'' & $\omega^{TK}_0$ & 
 $\omega^{cl}_0$ \\ \hline
 3  &   5.5  & 1.200 & 1.205 & 0.666 & 0.7563 \\ % 0.7585 \\
    &   5.6  & 1.230 & 1.235 & 0.666 & 0.7706 \\ % 0.7729 \\
    &   5.7  & 1.250 & 1.250 & 0.666 & 0.7806 \\ % 0.7806 \\
    &   5.8  & 1.265 & 1.266 & 0.666 & 0.7883 \\ % 0.7888 \\
\hline
 2  &   5.5  & 0.860 & 0.860 & 0.75  & 1.0389 \\ % 1.0389 \\
    &   5.6  & 0.945 & 0.943 & 0.75  & 1.0178 \\ % 1.0183 \\
    &   5.7  & 1.010 & 1.010 & 0.75  & 0.9962 \\ % 0.9962 \\
    &   5.8  & 1.080 & 1.081 & 0.75  & 0.9610 \\ % 0.9602 \\
\hline
 1  &   5.54 & 0.410 &  ---  & 1.0   & 1.3548 \\
 0  &   5.7  & 0.630 & 0.625 & 1.0   & 1.3063 \\ % 1.3079 \\
\hline
\end{tabular}
\caption{Comparison of the temporal clover coefficient, before tadpole
improving, from $\nu_0$-tuning. The $\nu_0$ are from Klassen's Table 4.
In the 4-th column we list $1/\nu$ from Klassen's Table 3. This should
be the same as $\nu_0$, at least for equivalent other parameters.}
\label{tab:nu0_tune}
\end{table}

It would, of course, be preferable to do the entire tuning non-perturbatively,
with the clover coefficients computed with Schr\"odinger functional
boundary conditions and appropriate chromoelectric (for $\omega_0$)
or chromomagnetic (for $\omega$) background fields. The above could then
be input values for the tuning.


\section*{Appendix}

Here we derive, following appendix B of Klassen, the classical dispersion
relation including a small background field.
For convenience, we consider here an action slightly more general than
(\ref{eq:Q_generic})
\bea
Q = m_0 + \nu_\mu \nabla_\mu \gamma_\mu -\half r_\mu a_\mu \Delta_\mu -
 \frac{1}{4} a \omega \cdot \sigma \cdot F ~.
\eea
with $\omega \cdot \sigma \cdot F \equiv 2 \sum{\mu < \nu} \omega_\mu
\sigma_{\mu\nu} F_{\mu\nu}$, $a_k=a$, $\nu_k=\nu$ and $\omega_k=\omega$ for
$k=1,2,3$, and summation over repeated indices implied.
The dispersion relation $Q Q^\dagger = 0$ then reads
\bea
0 = \nu_\mu \nu_\nu \gamma_\mu \gamma_\nu \nabla_\mu \nabla_\nu -
 (m_0 - \half r_\mu a_\mu \Delta_\mu -
 \frac{1}{4} a \omega \cdot \sigma \cdot F)^2 ~.
\eea
Using ${\bf p}$ for the spatial part of the 4-momentum $p$, and
\bea
\bar p_\mu = \frac{1}{a_\mu} \sin(a_\mu p_\mu) ~, \qquad
\hat p_\mu = \frac{2}{a_\mu} \sin \left( \frac{a_\mu p_\mu}{2} \right) ~,
\eea
we find, going to momentum space,
\bea
- \nu_0^2 \bar p_0^2 &=& \nu^2 \bar{\bf p}^2 - \half \nu_\mu \nu_\nu
 \sigma_{\mu\nu} F_{\mu\nu} + M({\bf p})^2 \nonumber \\
&& - M({\bf p})(-r_0 a_0 \hat p_0^2 + \half a \omega \cdot \sigma \cdot F)
 + \frac{1}{4} (-r_0 a_0 \hat p_0^2 + \half a \omega \cdot \sigma \cdot F)^2
\eea
with
\bea
M({\bf p}) = m_0 + \half a \hat{\bf p}^2 ~.
\eea
Next we use $\bar p^2_\mu = \hat p^2_\mu - \frac{1}{4} a^2_\mu \hat p^4_\mu$,
perform the Wick rotation to Minkowski space by introducing the energy $E$ via
\bea
\hat E^2 \equiv - \bar p^2_0 \equiv \left( \frac{2}{a_0}
 \sinh(a_0 E/2) \right)^2 ~,
\eea
and restrict to $\nu_0 = r_0$. Then
\bea
\hat E^2 \Big( 1 + \frac{a_0}{\nu_0} M({\bf p}) - \frac{a_0}{4\nu_0} a
 \omega \cdot \sigma \cdot F \Big) &=& \frac{\nu^2}{\nu_0^2} \bar{\bf p}^2
 + \frac{1}{\nu_0^2} M({\bf p})^2 - \\
&& \frac{1}{\nu_0^2} \sum{\mu < \nu} ( \nu_\mu \nu_\nu +a M({\bf p})
 \omega_\mu ) \sigma_{\mu\nu} F_{\mu\nu} + {\cal O}(F^2) ~. \nonumber
\eea
This looks like Klassen's eq. (B.6) but with all parameters $m_0$, $\nu$,
$r$ and $\omega_\mu$ replaced by $m_0/\nu_0$, $\nu/\nu_0$, $r/\nu_0$ and
$\omega_\mu/\nu_0$, implying also $M({\bf p}) \longrightarrow
M({\bf p})/\nu_0$.

In the absense of a background field we find
\bea
\hat E^2({\bf p}) = \frac{\bar{\bf p}^2 (\nu/\nu_0)^2 + (M({\bf p})/\nu_0)^2}{
 1 + a_0 M({\bf p})/\nu_0} ~.
\eea
In particular
\bea
\hat m^2 \equiv \hat E^2(0) = \frac{(m_0/\nu_0)^2}{1 + a_0 m_0/\nu_0} ~,
\eea
and the exact quark mass, or {\em pole} mass, is
\bea
m \equiv E(0) = \frac{1}{a_0} \log \Big( 1 + a_0 m_0/\nu_0 \Big) ~.
\eea

To do the tuning of the lattice dispersion relation, we note that from the
continuum dispersion relation $E^2 = m^2 + {\bf p}^2$ follows
\bea
\hat E^2 = \left( \frac{2}{a_0} \sinh(a_0 \sqrt{m^2 + {\bf p}^2} /2) \right)^2
 = \hat m^2 + \frac{\sinh(a_0 m)}{a_0 m} {\bf p}^2 + {\cal O}(a^2
 {\bf p}^4) ~. \nonumber
\eea
We compare this to the lattice dispersion relation expanded to the same order.
Setting $\nu_0 =1$ we can derive the tuning relations in (\ref{eq:nu_tune1}),
while setting $\nu = 1 = r$ we can derive (\ref{eq:nu0_tune1}).

On the other hand, for vanishing momentum we find
\bea
\hat E^2(F) ( 1 + a_0 m_0/\nu_0 ) &=& \left( \frac{m_0}{\nu_0} \right)^2 -
 \sum_{\mu < \nu} \left( \frac{\nu_\mu \nu_\nu}{\nu_0^2} +
 a \frac{m_0 \omega_\mu}{\nu_0^2} - \half a_0 a \hat m^2 \omega_\mu/\nu_0
 \right) \sigma_{\mu\nu} F_{\mu\nu} + {\cal O}(F^2) \nonumber \\
&=& \left( \frac{m_0}{\nu_0} \right)^2 - \sum_{\mu < \nu}
 \left( \frac{\nu_\mu \nu_\nu}{\nu_0^2} + \frac{a \omega_\mu}{a_0 \nu_0} 
 \sinh(a_0 m) \right) \sigma_{\mu\nu} F_{\mu\nu} + {\cal O}(F^2) ~.
\eea
In the continuum $E^2 = m^2 - \sum_{\mu < \nu} \sigma_{\mu\nu} F_{\mu\nu}$,
which leads to
\bea
\hat E^2 = \hat m^2 - \frac{\sinh(a_0 m)}{a_0 m} \sum_{\mu < \nu}
 \sigma_{\mu\nu} F_{\mu\nu} + {\cal O}(a^2 F^2) ~.
\eea
Therefore, for $\mu < \nu$ we need
\bea
\frac{\nu_\mu \nu_\nu}{\nu_0^2} + \frac{a \omega_\mu}{a_0 \nu_0}
 \sinh(a_0 m) = {\rm e}^{a_0 m} \frac{\sinh(a_0 m)}{a_0 m} ~.
\eea
Setting $\nu_0 =1$ we obtain the tuning relations in (\ref{eq:nu_tune2}),
while setting $\nu = 1 = r$ we can derive (\ref{eq:nu0_tune2}).


\end{document}
