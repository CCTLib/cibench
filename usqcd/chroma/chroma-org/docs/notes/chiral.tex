\magnification=1200
\baselineskip=18pt
The matrix $H+\gamma_5 m$ ( $m={\mu \over 1-\mu}$ ) 
can be written down in a block diagonal form
as follows:
\item{1.} Zero modes: $\pm m$
\item{2.} Edge modes: $\pm (1 + m)$ 
\item{3.} Non-zero pairs:
$$\pmatrix{ \lambda^2 + m & \lambda\sqrt{1-\lambda^2} \cr
\lambda\sqrt{1-\lambda^2} & -(\lambda^2 + m) \cr } $$

In the above form, the matrix can be inverted to give the following block
diagonal form for ${\tilde H}^{-1} = (1+2m)(H+\gamma_5 m)^{-1} - \gamma_5$ 
\item{1.} Zero modes: $\pm{1\over \mu}$
\item{2.} Edge modes: $\pm \mu$ 
\item{3.} Non-zero pairs:
$${1\over \lambda^2(1-\mu^2) + \mu^2}
\pmatrix{ \mu & (1-\mu^2)\lambda\sqrt{1-\lambda^2} \cr
(1-\mu^2)\lambda\sqrt{1-\lambda^2} & - \mu \cr } $$

It also gives the following block diagonal form for
${\tilde D}^{-1} = (1+2m)( D + m)^{-1} - 1$
\item{1.} Zero modes: ${1\over \mu}$
\item{2.} Edge modes: $\mu$ 
\item{3.} Non-zero pairs:
$${1\over \lambda^2(1-\mu^2) + \mu^2}
\pmatrix{ \mu & - (1-\mu^2)\lambda\sqrt{1-\lambda^2} \cr
(1-\mu^2)\lambda\sqrt{1-\lambda^2} &  \mu\cr } $$


Now we focus on SU(N) gauge theory with
two degenerate flavors in a physical volume $V$.
From the above expressions, we get
$$<\bar\psi_i\gamma_5\psi_i>_A = {1\over V} Tr[{\tilde H}^{-1}]= {2Q\over V}
\Bigl[{1\over \mu} -\mu\Bigr]$$ 
in a gauge field background with
a global topology of $Q$. 
This quantity goes to zero when averaged over
all gauge field configurations.
We also get
$$<\bar\psi_i \psi_i>_A = {1\over V} Tr[{\tilde D}^{-1}] = 
{2|Q|\over V}\Bigl[{1\over \mu}+ \mu \Bigr]+
{2\over V} \sum_i {2\mu\over \lambda_i^2(1-\mu^2) + \mu^2} $$
in a fixed gauge field background $A$. The sum is over all non-zero
eigenvalues $\lambda_i > 0 $ including the ones at $ 1$ that are
paired with $-1$. 
To compute the chiral condensate, we have to first take the infinite volume
limit and then take the massless limit. Since $<|Q|>$ is proportional to
$\sqrt{V}$ the first term goes to zero in the infinite volume limit.
We assume that the spectrum of non-zero eigenvalues is continuous in the
infinite volume limit with the spectral density per unit volume given by
$\rho(\lambda)$, a symmetric function in $\lambda$. We use the formula,
$$\int_0^\infty {1\over \lambda^2 + \mu^2} \rho(\lambda) d\lambda = {\pi \rho(0)\over
2\mu} + O(\mu) $$
With this formula, we have 
$$\lim_{\mu\rightarrow 0^+} \lim_{V\rightarrow\infty} <\bar\psi_i \psi_i> = 
2\pi \rho(0)$$

Consider the real O(4) vector made up of 
$\vec\phi = (\pi^a=
i\sum_{ij} \bar\psi_i(x)\gamma_5\tau^a_{ij}\psi_j(x) ,
\sigma=\sum_i\bar\psi_i(x)\psi_i(x))$. 

For the pseudo-scalar flavor triplet one gets,
$$\eqalign{\chi^a_P = & 
- \sum_{x,y,i,j} 
{1\over V} < \bar\psi_i(x)\gamma_5\tau^a_{ij}\psi_j(x) 
\bar\psi_k(y)\gamma_5\tau^a_{kl}\psi_l(y) >_A \cr
= &  {1\over V} Tr({\tilde H}^{-2}) \cr
= & - 8N + {2|Q|\over V} [\mu + {1\over \mu}]^2 + {2\over V} \sum_i 
{2(1+\mu^2) \over \lambda_i^2(1-\mu^2) + \mu^2} \cr 
= & - 8N +  [\mu+{1\over \mu}] <\bar\psi_i \psi_i>_A \cr } $$

For the scalar flavor singlet, one gets,
$$
\chi_{S}= \sum_{x,y,i,j}  
{1\over V} < \bar\psi_i(x)\psi_i(x) 
\bar\psi_j(y)\psi_j(y) >_A 
=  {1\over V} \Bigl[Tr({\tilde D}^{-1})\Bigr]^2 
- {1\over V} Tr({\tilde D}^{-2}) $$
$$\eqalign{
{1\over V} \Bigl[Tr({\tilde D}^{-1})\Bigr]^2 = &
{4Q^2\over V}\Bigl[{1\over\mu}+\mu\Bigr]^2 + {16|Q|\over V}\sum_i 
{(1+\mu^2) \over \lambda_i^2(1-\mu^2) + \mu^2} \cr 
& +{16\over V} \sum_{i,j} 
{\mu^2 \over [\lambda_i^2(1-\mu^2) + \mu^2][\lambda_j^2(1-\mu^2) + \mu^2]} 
\cr}
$$
$${1\over V} Tr({\tilde D}^{-2})
=  8N + {2|Q|\over V} [\mu - {1\over \mu}]^2 - {2\over V} \sum_i 
{2(1+\mu^2) \over \lambda_i^2(1-\mu^2) + \mu^2} 
+ {2\over V} \sum_i
{4\mu^2 \over [\lambda_i^2(1-\mu^2) + \mu^2]^2}
$$

With these equations in place, we compute
the O(4) invariant correlator,
$$\eqalign{ \sum_{x,y}
{1\over V} <\vec\phi(x)\cdot \vec\phi(y)>_A = &
-32N + {4Q^2\over V}\Bigl[{1\over\mu}+\mu\Bigr]^2 
+ {4|Q|\over V}\Bigl[{1\over\mu^2}+\mu^2+4\Bigr] 
 + {16|Q|\over V}\sum_i 
{(1+\mu^2) \over \lambda_i^2(1-\mu^2) + \mu^2} \cr 
& + {16\over V} \sum_i {(1+\mu^2) \over \lambda_i^2(1-\mu^2) + \mu^2} 
 - {8\over V} \sum_i
{\mu^2 \over [\lambda_i^2(1-\mu^2) + \mu^2]^2} \cr
& +{16\over V} \sum_{i,j} 
{\mu^2 \over [\lambda_i^2(1-\mu^2) + \mu^2][\lambda_j^2(1-\mu^2) + \mu^2]}
\cr }
$$
In the massless limit at fixed lattice volume, the above equation
reduces to
$$\lim_{\mu\rightarrow 0}\sum_{x,y}
{1\over V} <\vec\phi(x)\cdot \vec\phi(y)> = 
\lim_{\mu\rightarrow 0} {4<Q^2+|Q|>\over \mu^2 V}
+{16\over V} < \sum_i {1\over\lambda_i^2} > -32N
$$
where the average is now over an ensemble of gauge fields.

Consider the other real O(4) vector made up of 
$\vec\rho = (\tilde\pi^a=
-\sum_{ij} \bar\psi_i(x)\tau^a_{ij}\psi_j(x) ,
\tilde\sigma=i\sum_i\bar\psi_i(x)\gamma_5\psi_i(x))$. 

For the scalar flavor triplet one gets,
$$\eqalign{\chi^a_S = & 
\sum_{x,y,i,j}   
{1\over V} < \bar\psi_i(x)\tau^a_{ij}\psi_j(x) 
\bar\psi_k(y)\tau^a_{kl}\psi_l(y) >_A 
=  - {1\over V} Tr({\tilde D}^{-2}) \cr
& = - 8N - {2|Q|\over V} [\mu - {1\over \mu}]^2 + {2\over V} \sum_i 
{2(1+\mu^2) \over \lambda_i^2(1-\mu^2) + \mu^2} 
- {2\over V} \sum_i
{4\mu^2 \over [\lambda_i^2(1-\mu^2) + \mu^2]^2}\cr 
& = -8N +(1-\mu^2) {d\over d\mu} <\bar\psi_i \psi_i>_A \cr
}
$$

For the 
pseudo-scalar flavor singlet, we get,
$$\eqalign{ \chi_P = &
-\sum_{x,y} 
{1\over V} < \bar\psi_i(x)\gamma_5\psi_i(x) \bar\psi_j(y)\gamma_5\psi_j(y) >_A \cr
= & 
-{1\over V} \Bigl[Tr({\tilde H}^{-1})\Bigr]^2 
+ {1\over V} Tr({\tilde H}^{-2}) \cr
= & -{4Q^2\over V} \Bigl[{1\over \mu}
- \mu\Bigr]^2 -8N + {2|Q|\over V} [\mu + {1\over \mu}]^2 + {2\over V} \sum_i 
{2(1+\mu^2) \over \lambda_i^2(1-\mu^2) + \mu^2} \cr 
= & - {4Q^2\over V} \Bigl[{1\over \mu}
-\mu\Bigr]^2 - 8N + [\mu+{1\over \mu}] <\bar\psi_i \psi_i>_A \cr } $$

With these equations in place, we compute
the other O(4) invariant correlator,
$$\eqalign{ \sum_{x,y}
{1\over V} <\vec\rho(x)\cdot \vec\rho(y)>_A = &
- 32N - {4Q^2\over V}\Bigl[{1\over\mu}-\mu\Bigr]^2 
-{4|Q|\over V}\Bigl[{1\over\mu^2}+\mu^2-4\Bigr] 
\cr
& +{16\over V} \sum_i
{1+ \mu^2 \over \lambda_i^2(1-\mu^2) + \mu^2}
- {24\over V} \sum_i
{\mu^2 \over [\lambda_i^2(1-\mu^2) + \mu^2]^2}
\cr }
$$
In the massless limit at fixed lattice volume, the above equation
reduces to
$$\lim_{\mu\rightarrow 0}\sum_{x,y}
{1\over V} <\vec\rho(x)\cdot \rho(y)> = 
{16\over V} < \sum_i {1\over\lambda_i^2} >  
- \lim_{\mu\rightarrow 0}{4<Q^2+|Q|>\over \mu^2 V} - 32N
$$
where the average is now over an ensemble of gauge fields.

Combining the two different O(4) correlators one gets,
$$\eqalign{& \sum_{x,y}
 {1\over V}   \Bigl[ <\vec\phi(x)\cdot\vec\phi(y)>_A- 
<\vec\rho(x)\cdot\vec\rho(y)>_A \Bigr] \cr
& = {8Q^2\over V}\Bigl[{1\over\mu^2}+\mu^2\Bigr] 
+{8|Q|\over V}\Bigl[{1\over\mu^2}+\mu^2\Bigr] 
+ {16|Q|\over V}\sum_i 
{(1+\mu^2) \over \lambda_i^2(1-\mu^2) + \mu^2} \cr
& + {16\over V} \sum_i
{\mu^2 \over [\lambda_i^2(1-\mu^2) + \mu^2]^2}
+{16\over V} \sum_{i,j} 
{\mu^2 \over [\lambda_i^2(1-\mu^2) + \mu^2][\lambda_j^2(1-\mu^2) + \mu^2]}
\cr }
$$
In the massless limit, the above combination reduces to
$$ \lim_{\mu\rightarrow 0} \sum_{x,y}
 {1\over V}   \Bigl[ <\vec\phi(x)\cdot\vec\phi(y)>- 
<\vec\rho(x)\cdot\vec\rho(y)> \Bigr] = 
\lim_{\mu\rightarrow 0}{8<Q^2+|Q|>\over \mu^2 V} $$

We also note that
$$\eqalign{ 
\omega^a & = \chi^a_P - \chi^a_S \cr
& = {4|Q|\over V} \Bigl[\mu^2 + {1\over \mu^2}\Bigr] +
 {2\over V} \sum_i
{4\mu^2 \over [\lambda_i^2(1-\mu^2) + \mu^2]^2} \cr
& = \Bigl[\mu + {1\over \mu}\Bigr] <\bar\psi_i\psi_i>_A 
-\Bigl[1 -\mu^2 \Bigr] {d\over d\mu} <\bar\psi_i\psi_i>_A \cr}
$$

In finite volume, chiral symmetry in not broken in the massless limit
and therefore:
$$
\lim_{\mu\rightarrow 0} \sum_{x,y}
{1\over V} <\phi^a(x)\phi^a(y)> =
 \lim_{\mu\rightarrow 0} \sum_{x,y}
{1\over V} <\phi^b(x)\phi^b(y)>$$
for all $a$ and $b$. Setting $a=1$ and $b=4$
yields
$$\lim_{\mu\rightarrow 0} {1\over V} Tr{\tilde H}^{-2} =
\lim_{\mu\rightarrow 0} {1\over V} [Tr{\tilde D}^{-1}]^2 - Tr{\tilde D}^{-2}
$$
Similarly,
$$
\lim_{\mu\rightarrow 0} \sum_{x,y}
{1\over V} <\rho^a(x)\rho^a(y)> =
 \lim_{\mu\rightarrow 0} \sum_{x,y}
{1\over V} <\rho^b(x)\rho^b(y)>$$
for all $a$ and $b$. Setting $a=1$ and $b=4$
yields
$$\lim_{\mu\rightarrow 0} {1\over V} Tr{\tilde D}^{-2} =
\lim_{\mu\rightarrow 0} {1\over V} [Tr{\tilde H}^{-1}]^2 - Tr{\tilde H}^{-2}$$
Combining the two equations one has
$$\lim_{\mu\rightarrow 0} {1\over V} [Tr{\tilde H}^{-1}]^2 =
\lim_{\mu\rightarrow 0} {1\over V} [Tr{\tilde D}^{-1}]^2$$
which is satisfied if
$$ \lim_{\mu\rightarrow 0} {1\over V} < |Q| \sum_i {1\over \lambda_i^2} > =0
$$
as we have assumed before.
Each one of the relations indvidually yields,
$$\lim_{\mu\rightarrow 0} {<|Q|>\over \mu^2 V} = 
\lim_{\mu\rightarrow 0} {<Q^2>\over \mu^2 V}$$


\end
