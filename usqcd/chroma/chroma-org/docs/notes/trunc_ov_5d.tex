\documentclass[12pt]{article}
\usepackage{amsmath}

% Somewhat wider and taller page than in art12.sty
\topmargin -0.4in  \headsep 0.0in  \textheight 9.0in
\oddsidemargin 0.25in  \evensidemargin 0.25in  \textwidth 6.5in

\footnotesep 14pt
\floatsep 28pt plus 2pt minus 4pt      % Nominal is double what is in art12.sty
\textfloatsep 40pt plus 2pt minus 4pt
\intextsep 28pt plus 4pt minus 4pt

\begin{document}

\newcommand{\half}{\frac{1}{2}}
%\newcommand{\be}{\begin{equation}}
%\newcommand{\ee}{\end{equation}}
\newcommand{\be}{\begin{displaymath}}
\newcommand{\ee}{\end{displaymath}}
\newcommand{\bea}{\begin{eqnarray}}
\newcommand{\eea}{\end{eqnarray}}
\newcommand{\bdm}{\begin{displaymath}}
\newcommand{\edm}{\end{displaymath}}
\newcommand{\<}{\langle}
\renewcommand{\>}{\rangle}
\newcommand{\Tr}{\mbox{Tr}}

\centerline{\bf \Large Borici's truncated overlap}
\vskip 5mm

This document describes Borici's construction of a 5-d domain wall fermion
like approach to Neuberger's polar decomposition approximation of the
overlap Dirac operator. It uses my conventions throughout.

The usual, 4-d, Wilson-Dirac operator is given by
\bea
D_w(M) \!\! &=& \!\! (4+M) \delta_{x,y} - \frac{1}{2} \sum_{\mu=1}^4 \Bigl[
 (1 - \gamma_\mu) U_\mu(x) \delta_{x+\mu,y} +
 (1 + \gamma_\mu) U_\mu^\dagger(y) \delta_{x,y+\mu} \Bigr] \nonumber \\
\!\! &=& \!\! \begin{pmatrix} B + M & C \\ -C^\dagger & B + M \end{pmatrix} .
\label{eq:D_w}
\eea
Here, with $\sigma_\mu = (\sigma_k, i {\bf 1})$,
\bea
C \!\! &=& \!\! \frac{1}{2} \sum_{\mu=1}^4 \sigma_\mu \Bigl[ U_\mu(x)
 \delta_{x+\mu,y} - U_\mu^\dagger(y) \delta_{x,y+\mu} \Bigr] , \nonumber \\
B \!\! &=& \!\! \frac{1}{2} \sum_{\mu=1}^4 \Bigl[ 2 \delta_{x,y} -  U_\mu(x)
 \delta_{x+\mu,y} - U_\mu^\dagger(y) \delta_{x,y+\mu} \Bigr] .
\label{eq:C_B}
\eea
We are using a chiral basis: $\gamma_\mu = \begin{pmatrix} 0 & \sigma_\mu \\
\sigma_\mu^\dagger & 0 \end{pmatrix}$, $\gamma_5 = \begin{pmatrix} 1 & 0 \\
0 & -1 \end{pmatrix}$. We will also use the hermitian Wilson-Dirac operator
\bea
H_w(M) = \gamma_5 D_w(M) = \begin{pmatrix} B + M & C \\
	C^\dagger & -B - M \end{pmatrix} .
\label{eq:H_w}
\eea
Usually, we will use a large negative mass here, but often omit the
argument from $D_w$ and $H_w$.

In this notation, the usual domain wall fermion action of Shamir reads
with an explicit fifth dimensional lattice spacing
(e.g. hep-lat/9807012) for the hopping term but keeping a dimensional mass
term
\bea
S_{DW} = - \bar \Psi D^{(5)}_{DW} \Psi = - \sum_{i=1}^N \bar \Psi_i
 \Bigl\{ \bigl[a_5 D_w(-M) + 1 \bigr] \Psi_i - P_- \Psi_{i+1} -
 P_+ \Psi_{i-1} \Bigr\}
\label{eq:S_DW}
\eea
with boundary conditions in the extra (fifth) direction:
\bea
P_- \Psi_{N+1} = -m P_- \Psi_1 ,  \qquad P_+ \Psi_0 = -m P_+ \Psi_N .
\label{eq:bc}
\eea
%
$P_\pm$ are the chiral projectors, $P_\pm = \frac{1}{2} ( 1 \pm \gamma_5)$.
$m \le 1$ is proportional to the quark mass.
The 4-d (light) fermion degrees of freedom are identified with
\bea
q^R = P_+ \Psi_N = \Psi_N^R ,&& \quad  q^L = P_- \psi_1 = \Psi_1^L ,
 \nonumber \\
\bar q^R = \bar \Psi_N P_- = \bar \Psi_N^R ,&& \quad \bar q^L =
\bar \psi_1 P_+ = \bar \Psi_1^L .
\label{eq:light}
\eea

Borici's 5-d truncated overlap action is then
\bea
S_B = - \bar \Psi D^{(5)}_B \Psi = - \sum_{i=1}^N \bar \Psi_i
 \!\!\!\! && \!\!\!\! \Bigl\{ \bigr[a_5 D_w(-M) + 1 \bigl] \Psi_i +
 \bigr[a_5 D_w(-M) - 1 \bigl] P_- \Psi_{i+1} + \nonumber \\
 \!\!\!\! && \!\!\! \bigr[a_5 D_w(-M) - 1 \bigl] P_+ \Psi_{i-1} \Bigr\}
 + \bar \Psi_1 \hat A(m) \bigr[ P_- \Psi_1 + P_+ \Psi_N \bigl]
\label{eq:S_B}
\eea
with unchanged boundary conditions in the extra direction.
Here we have included an as yet unspecified quark mass dependent
4-d operator $\hat A(m)$, which will prove useful later on.
The kernel of the 5-d operator is, including the boundary conditions,
\bea
D^{(5)}_{B} = \begin{pmatrix}
D_+ - \hat{A}P_- & D_-P_- & 0 & 0&\cdots & 0 & 0 &
-m D_-P_+ - \hat{A} P_+ \cr
D_-P_+ & D_+ & D_-P_- & 0&\cdots & 0 & 0 & 0 \cr
0 & D_-P_+ & D_+ & D_-P_- &\cdots & 0 & 0 & 0 \cr
\vdots & \vdots & \vdots & \vdots & \vdots & \vdots & \vdots & \vdots \cr
0 & 0 & 0 & 0 & \cdots & D_-P_+ & D_+ & D_-P_- \cr
-m D_-P_- & 0 & 0 & 0 & \cdots & 0 & D_-P_+ & D_+ \cr
\end{pmatrix}
\label{eq:D_5d_matrix}
\eea
where
\bea
D_\pm = a_5 D_w(-M) \pm 1 ~.
\eea

What follows are manipulations that show the relation between Borici's
domain wall version, and its associated pseudo-fermions, to the truncated
overlap (the polar decomposition approximation to the overlap) of
Neuberger.

Define ${\cal P}$ by
\bea
{\cal P}_{jk} = \begin{cases}
 P_- \delta_{j,k} + P_+ \delta_{j+1,k} & \text{for $j = 1, \dots, N-1$} \\
 P_- \delta_{N,k} + P_+ \delta_{1,k} & \text{for $j = N$} \end{cases} .
\eea
This has an inverse ${\cal P}^{-1} = {\cal P}^\dagger$, given by
\bea
{\cal P}^{-1}_{jk} = \begin{cases}
 P_- \delta_{j,k} + P_+ \delta_{j-1,k} & \text{for $j = 2, \dots, N$} \\
 P_- \delta_{1,k} + P_+ \delta_{N,k} & \text{for $j = 1$} \end{cases} .
\eea

Now, introduce $\chi_i$'s through $\Psi_i = ({\cal P} \chi)_i$. Then, we can
rewrite Borici's domain wall action as
\bea
S_B = \!\! &-& \!\! \Bigl\{ \bar \Psi_1 \gamma_5 \bigl[ (a_5 H_w -1)
 (P_- -m P_+) \chi_1 - \gamma_5 \hat A \chi_1 + (a_5 H_w +1) \chi_2 \bigr]
 \nonumber \\
\label{eq:S_B2}
 \!\! &+& \!\! \quad \sum_{i=2}^{N-1} \bar \Psi_i \gamma_5
 \bigl[ (a_5 H_w -1) \chi_i + (a_5 H_w +1) \chi_{i+1} \bigr] \\
 \!\! &+& \!\! \quad \bar \Psi_N \gamma_5 \bigl[ (a_5 H_w -1) \chi_N
  + (a_5 H_w +1) (P_+ -m P_-) \chi_1 \bigr] \Bigr\} . \nonumber
\eea
To arrive at this we have used $\gamma_5 P_+ = P_+$ and $\gamma_5 P_- = - P_-$,
as well as the boundary conditions on the fermion fields.

Next we introduce $\bar \Psi_i = \bar \chi_i (a_5 H_w -1)^{-1} \gamma_5$ and
\bea
T^{-1} = \frac{1 + a_5 H_w}{1 - a_5 H_w} ~.
\label{eq:T}
\eea
This change of variables would give rise to a Jacobian in a dynamical
simulation. However, it would be exactly canceled by the Jacobian of
this transformation for the pseudo-fermion fields, whose action is
obtained by the replacement $m \to 1$, since integration of
fermions (Grassman fields) acts like differentiation.

Anyway, Borici's domain wall action reads now
\bea
S_B = \!\! &-& \!\! \biggl\{ \bar \chi_1 \bigl[ (P_- -m P_+) \chi_1 -
 (a_5 H_w -1)^{-1} \gamma_5 \hat A \chi_1 - T^{-1} \chi_2 \bigr] \nonumber \\
 \!\! &+& \!\! \quad \sum_{i=2}^{N-1} \bar \chi_i \bigl[ \chi_i-
 T^{-1} \chi_{i+1} \bigr]
 + \bar \chi_N \bigl[ \chi_N - T^{-1} (P_+ -m P_-) \chi_1 \bigr] \biggr\} .
\label{eq:S_B3}
\eea

Now, we integrate out, in succession, $\chi_N, \bar \chi_N$, $\chi_{N-1},
\bar \chi_{N-1}$, $\dots$, $\chi_2, \bar \chi_2$. For this we use, at
the first step,
\bea
 \!\! && \!\! \bar \chi_N \chi_N - \bar \chi_{N-1} T^{-1} \chi_N -
 \bar \chi_N T^{-1} (P_+ -m P_-) \chi_1 = \nonumber \\
 \!\! && \!\! \bigl[ \bar \chi_N - \bar \chi_{N-1} T^{-1} \bigr] \bigl[ \chi_N -
 T^{-1} (P_+ -m P_-) \chi_1 \bigr] - \bar \chi_{N-1} T^{-2} (P_+ -m P_-) \chi_1
\nonumber
\eea
and at the $(N-i)$-th step
\bea
\label{eq:int_i}
 \!\!\!\! && \!\!\!\! \bar \chi_{i+1} \chi_{i+1} - \bar \chi_i T^{-1}
 \chi_{i+1} - \bar \chi_{i+1} T^{-N+i} (P_+ -m P_-) \chi_1 = \\
 \!\!\!\! && \!\!\!\! \bigl[ \bar \chi_{i+1} - \bar \chi_i T^{-1} \bigr]
 \bigl[ \chi_{i+1} - T^{-N+i} (P_+ -m P_-) \chi_1 \bigr] -
 \bar \chi_i T^{-N+i-1} (P_+ -m P_-) \chi_1 . \nonumber
\eea
With a change of variables $\chi_{i+1}^\prime = \chi_{i+1} - T^{-N+i}
(P_+ -m P_-) \chi_1$ and $\bar \chi_{i+1}^\prime = \bar \chi_{i+1} -
\bar \chi_i T^{-1}$ the integration over $\chi_{i+1}^\prime,
\bar \chi_{i+1}^\prime$ is trivial, giving a factor 1.
At the end, we arrive at the 4-d action for $\chi_1, \bar \chi_1$
\bea
S^{(4)}_B = - \bar \chi_1 \bigl[ (P_- -m P_+) - T^{-N} (P_+ -m P_-)
 - (a_5 H_w -1)^{-1} \gamma_5 \hat A \bigr] \chi_1
 = - \bar \chi_1 D^{(4)}_B(m) \chi_1 .
\label{eq:S_B4}
\eea
The kernel is
\bea
D^{(4)}_B(m) &=& (P_- -m P_+) - T^{-N} (P_+ -m P_-) - (a_5 H_w -1)^{-1}
 \gamma_5 \hat A \nonumber \\
\label{eq:D_B4}
 &=& -\left[ \frac{1+m}{2} \left(T^{-N} + 1 \right) \gamma_5 +
 \frac{1-m}{2} \left(T^{-N} -1 \right) +
 (a_5 H_w -1)^{-1} \gamma_5 \hat A \right] \\
 &=& \Bigl[ \left(T^{-N} + 1 \right) \gamma_5 \Bigr] \times
 \left[ \frac{1+m}{2} + \frac{1-m}{2} \gamma_5 \frac{T^{-N} -1}{T^{-N} + 1} +
 \gamma_5 \frac{1}{(T^{-N} + 1)(a_5 H_w -1)} \gamma_5 \hat A \right] ~.
 \nonumber
\eea

Finally, integrating out $\chi_1, \bar \chi_1$ and dividing by the
pseudo-fermion determinant, obtained by the substitution $m \to 1$
and requiring $\hat A(m=1) = 0$, we find
\bea
\frac{\det D^{(5)}_B(m)}{\det D^{(5)}_B(1)}
 = \frac{\det D^{(4)}_B(m)}{\det D^{(4)}_B(1)}
 = \det \left\{ \left[ D^{(4)}_B(1) \right]^{-1} D^{(4)}_B(m) \right\} .
\label{eq:5d_4d_det}
\eea
Now, from eq.~(\ref{eq:D_B4}), $D^{(4)}_B(1) = \left(T^{-N} + 1 \right)
\gamma_5$ and thus we get
\bea
\left[ D^{(4)}_B(1) \right]^{-1} D^{(4)}_B(m) =
 \frac{1}{2} \Biggl[1+m + (1-m) \gamma_5 \frac{T^{-N} -1}{T^{-N} + 1}
 + \gamma_5 \frac{2}{(T^{-N} + 1)(a_5 H_w -1)} \gamma_5 \hat A \Biggr] ~.
\label{eq:D_T}
\eea
Noting that
\bea
\frac{T^{-N} -1}{T^{-N} + 1} = \frac{(1+a_5 H_w)^N - (1-a_5 H_w)^N}
 {(1+a_5 H_w)^N + (1-a_5 H_w)^N} = \varepsilon_{N/2}(a_5 H_w)
\eea
where $\varepsilon_n(x)$ is Neuberger's polar decomposition approximation
to $\epsilon(x)$, we see that, ignoring the $\hat A$ term for the moment,
\bea
\left[ D^{(4)}_B(1) \right]^{-1} D^{(4)}_B(m) = D_{TOV}(m) =
\frac{1}{2} \Bigl[1+m + (1-m) \gamma_5 \varepsilon_{N/2}(a_5 H_w) \Bigr] .
\label{eq:D_TOV}
\eea
This is just Neuberger's "truncated overlap"! For $N \to \infty$ it goes into
the usual overlap.

We now can use $\hat A$ to project out low-lying Wilson eigenvectors, $v_i$,
for which $\varepsilon_{N/2}$ with finite $N$ is not a sufficiently accurate
approximation to $\epsilon(x)$. Let
\bea
H_w v_i = \lambda_i v_i, \qquad T v_i = T_i v_i, \qquad
\hat{P}_i = v_i v_i^\dagger ~,
\eea
where, from eq.~(\ref{eq:T}), 
\bea
T_i = \frac{1 - a_5 \lambda_i}{1 + a_5 \lambda_i} ~.
\label{eq:T_i}
\eea
The projection can then be achieved by setting
\bea
\hat A(m) = (1-m) \gamma_5 \sum_i f_i \hat{P}_i ,
\eea
with
\bea
f_i &=& \frac{1}{2}(a_5\lambda_i -1) \left[-\left(T_i^{-N} -1 \right) +
 \left(T_i^{-N} + 1\right)\epsilon(a_5 \lambda_i)\right]\nonumber\\
  &=& \left\{\begin{array}{cc}
       (a_5\lambda_i -1)&\forall\lambda_i > 0\\
       \frac{(1 + a_5\lambda_i)^N}{(1- a_5\lambda_i)^{N-1}}&
       \forall\lambda_i < 0\\
\end{array}\right\}
\label{eq:f_i}
\eea
Note that $\hat A(m)$ vanishes for $m=1$, {\it i.e.} for the pseudo-fermions.
With this $\hat A(m)$ we find, instead of (\ref{eq:D_TOV}),
\bea
&& \left[ D^{(4)}_B(1) \right]^{-1} D^{(4)}_B(m) = D_{ov}(m) = \nonumber \\
&& \qquad \frac{1}{2} \biggl\{1+m + (1-m) \gamma_5 \Bigl[
 \varepsilon_{N/2}(a_5 H_w) \Bigl(1 - \sum_i \hat{P}_i \Bigr) +
 \sum_i \epsilon(a_5 \lambda_i) \hat{P}_i \Bigr] \biggr\} ~.
\label{eq:D_ov_proj}
\eea

Our next step is to relate the 4-d overlap propagator to the
5-d propagator of Borici's version of the domain wall fermion action. In all
steps below $\hat A(m)$ for the eigenvalue projections is included. From
(\ref{eq:D_ov_proj}) we find, obviously,
\bea
D_{ov}^{-1}(m) = \left[ D^{(4)}_B(m) \right]^{-1} D^{(4)}_B(1) .
\label{eq:D_ov_inv}
\eea
To connect this to the 5-d theory, we consider, motivated by the fact
that the light 4-d fermion $q = ({\cal P}^{-1} \Psi)_1 = \chi_1$,
\bea
X \!\! &=& \!\! \left\{ {\cal P}^{-1} \left[ D^{(5)}_B(m) \right]^{-1}
 D^{(5)}_B(1) {\cal P} \right\}_{11} \nonumber \\
\label{eq:X}
 \!\! &=& \!\! \frac{1}{Z} \int \prod_i d\Psi_i d\bar \Psi_i \sum_k
 ({\cal P}^{-1} \Psi)_1 \bar \Psi_k \left[D^{(5)}_B(1) {\cal P} \right]_{k1}
 {\rm e}^{-S_B} \\
 \!\! &=& \!\! \frac{1}{Z^\prime} \int \prod_i d\chi_i d\bar \chi_i \sum_k
 \chi_1 \bar \chi_k (a_5 H_w -1)^{-1} \gamma_5 \left[D^{(5)}_B(1)
 {\cal P} \right]_{k1} {\rm e}^{-S_B} \nonumber \\
 \!\! &=& \!\! \frac{1}{Z^\prime} \int \prod_i d\chi_i d\bar \chi_i \sum_k
 \chi_1 \bar \chi_k \left[\tilde D^{(5)}_B(1) \right]_{k1} {\rm e}^{-S_B} .
 \nonumber
\eea
Here we have introduced $\tilde D^{(5)}_B = (a_5 H_w -1)^{-1} \gamma_5 D^{(5)}_B
{\cal P}$ such that the 5-d action is $S_B = - \bar \chi \tilde D^{(5)}_B
\chi$. Now, we integrate out, in succession, $\chi_N, \bar \chi_N$,
$\chi_{N-1}, \bar \chi_{N-1}$, $\dots$, $\chi_2, \bar \chi_2$. From the
transformations (\ref{eq:int_i}) we see that $\bar \chi_k \rightarrow
\bar \chi_1 T^{-k+1}$ in the process. Therefore, we obtain
\bea
X = \frac{1}{Z^\prime} \int d\chi_1 d\bar \chi_1 \chi_1 \bar \chi_1
 \sum_k T^{-k+1} \left[\tilde D^{(5)}_B(1) \right]_{k1}
 {\rm e}^{\bar \chi_1 D^{(4)}_B(m) \chi_1} .
\eea
But from (\ref{eq:S_B3}) we see that
\bea
\sum_k T^{-k+1} \left[\tilde D^{(5)}_B(1) \right]_{k1} = (P_- - P_+) -
 T^{-N} (P_+ - P_-) = D^{(4)}_B(1) ~,
\label{eq:pseudo_term}
\eea
where we used the fact that $\hat A(1) = 0$.
Thus we finally obtain
\bea
X \!\! &=& \!\! \left\{ {\cal P}^{-1} \left[ D^{(5)}_B(m) \right]^{-1}
 D^{(5)}_B(1) {\cal P} \right\}_{11} \nonumber \\
 \!\! &=& \!\! \left[ D^{(4)}_B(m) \right]^{-1} D^{(4)}_B(1)
 = D_{ov}^{-1}(m) ~.
\label{eq:D_ov5d_inv}
\eea

To solve
\bea
D_{ov}(m) \psi = b
\label{eq:d_prop}
\eea
we introduce $\tilde b = (b,0,\dots,0)^T$ and solve
\bea
D^{(5)}_B(m) \phi = D^{(5)}_B(1) {\cal P} \tilde b .
\label{eq:5d_prop}
\eea
$\psi$ is then obtained as
\bea
\psi = \left( {\cal P}^{-1} \phi \right)_1 .
\label{eq:prop_soln}
\eea

So far, we have related $D_{ov}^{-1}(m)$ and $\det D_{ov}(m)$, or their
truncated versions, to 5-d operators, see (\ref{eq:5d_4d_det}),
(\ref{eq:D_TOV}) and (\ref{eq:D_ov_proj}) and finally (\ref{eq:D_ov5d_inv}).
{\it E.g.} for eigenvalue calculations it would be useful to establish
a similar connection for $D_{ov}(m)$ or $D_{TOV}(m)$. Consider
$\tilde D^{(5)}_B(m)$ introduced in (\ref{eq:X}). We find
\bea
&& \left[ \tilde D^{(5)}_B(m) - \tilde D^{(5)}_B(1) \right]_{ij} = \\
 && \left[ (1-m) P_+ - (a_5 H_w -1)^{-1} \gamma_5 \hat A(m) \right]
 \delta_{i1} \delta_{j1} - T^{-1} (1-m) P_- \delta_{iN} \delta_{j1} ~.
 \nonumber
\eea
Thus
\bea
&& \left[ \tilde D^{(5)}_B(1) \right]^{-1} \tilde D^{(5)}_B(m) = \\
 && 1 + \left[ \tilde D^{(5)}_B(1) \right]^{-1}_{i1}
 \left[ (1-m) P_+ - (a_5 H_w -1)^{-1} \gamma_5 \hat A(m) \right] \delta_{j1}
 - (1-m) \left[ \tilde D^{(5)}_B(1) \right]^{-1}_{iN} T^{-1} P_- \delta_{j1} ~.
 \nonumber
\eea
So only the first column is different from the 5-d unit matrix. Let's
call the entries in the first column $X_1, X_2, \dots , X_N$.
Similarly, its inverse is
\bea
&& \left[ \tilde D^{(5)}_B(m) \right]^{-1} \tilde D^{(5)}_B(1) = \\
 && 1 - \left[ \tilde D^{(5)}_B(m) \right]^{-1}_{i1}
 \left[ (1-m) P_+ - (a_5 H_w -1)^{-1} \gamma_5 \hat A(m) \right] \delta_{j1}
 + (1-m) \left[ \tilde D^{(5)}_B(m) \right]^{-1}_{iN} T^{-1} P_- \delta_{j1} ~.
 \nonumber
\eea
Again, only the first column is different from the 5-d unit matrix. Let's
call the entries in the first column $Y_1, Y_2, \dots , Y_N$.
Since these 5-d matrices are the inverse of each other, their product
is the 5-d unit matrix, {\it i.e.}
\bea
Y_1 X_1 = 1  &\Rightarrow&  Y_1 = (X_1)^{-1} \nonumber \\
Y_2 X_1 + X_2 = 0  &\Rightarrow&  Y_2 = - X_2 (X_1)^{-1} \nonumber \\
\vdots   & & \vdots \\
Y_j X_1 + X_j = 0  &\Rightarrow&  Y_j = - X_j (X_1)^{-1} \quad {\text for
j=1, \dots, N}
\eea
The first equation, $Y_1 = (X_1)^{-1}$ establishes the relation we are looking
for
\bea
D_{ov}(m) = \left\{ {\cal P}^{-1} \left[ D^{(5)}_B(1) \right]^{-1}
 D^{(5)}_B(m) {\cal P} \right\}_{11} ~.
\eea

The hermitian conjugate of the 5-d operator (\ref{eq:D_5d_matrix}) is
different than for the usual DWF operator.
In particular there is no (generalized) $\gamma_5$ hermiticity:
\bea
D^{(5)\dagger}_{B} = \begin{pmatrix}
D_+^\dagger - P_-\hat{A}^\dagger & P_+D_-^\dagger & 0 & 0&\cdots & 0 & 0 &
-m P_- D_-^\dagger \cr
P_- D_-^\dagger & D_+^\dagger & P_+ D_-^\dagger & 0&\cdots & 0 & 0 & 0 \cr
0 & P_- D_-^\dagger & D_+^\dagger & P_+ D_-^\dagger &\cdots & 0 & 0 & 0 \cr
\vdots & \vdots & \vdots & \vdots & \vdots & \vdots & \vdots & \vdots \cr
0 & 0 & 0 & 0 & \cdots &  P_- D_-^\dagger & D_+^\dagger & P_+ D_-^\dagger \cr
-m P_+ D_-^\dagger - P_+\hat{A}^\dagger & 0 & 0 & 0 & \cdots & 0 & 
P_- D_-^\dagger & D_+^\dagger \cr
\end{pmatrix}
\label{eq:Ddag_5d_matrix}
\eea
where
\be
D^\dagger_\pm = a_5 D_w^\dagger(-M) \pm 1,\qquad
\hat{A}^\dagger(m) = (1-m)\sum_i f_i\hat{P}_i \gamma_5\ .
\ee

We can go through manipulations similar to those that let from the 5-d
operator (\ref{eq:D_5d_matrix}) to the 4-d (truncated) overlap operator
(\ref{eq:D_TOV}) for the usual domain wall fermions of (\ref{eq:S_DW}).
The kernel of the 5-d operator is, including the boundary conditions
and, as in (\ref{eq:D_5d_matrix}), an as yet unspecified quark mass
dependent operator $\hat A(m)$,
\bea
D^{(5)}_{DW} = \begin{pmatrix}
D_+ - \hat{A}P_- & -P_- & 0 & 0&\cdots & 0 & 0 & m P_+ - \hat{A} P_+ \cr
-P_+ & D_+ & -P_- & 0&\cdots & 0 & 0 & 0 \cr
0 & -P_+ & D_+ & -P_- &\cdots & 0 & 0 & 0 \cr
\vdots & \vdots & \vdots & \vdots & \vdots & \vdots & \vdots & \vdots \cr
0 & 0 & 0 & 0 & \cdots & -P_+ & D_+ & -P_- \cr
m P_- & 0 & 0 & 0 & \cdots & 0 & -P_+ & D_+ \cr
\end{pmatrix}
\label{eq:D_5d_DW}
\eea
Setting $\Psi_i = ({\cal P} \chi)_i$ as before, we can rewrite the domain
wall action (\ref{eq:S_DW}) as
\bea
S_{DW} = \!\! &-& \!\! \Bigl\{ \bar \Psi_1 \gamma_5 \bigl[ (a_5 H_w P_- -1)
 (P_- -m P_+) \chi_1 - \gamma_5 \hat A \chi_1 + (a_5 H_w P_+ +1) \chi_2 \bigr]
 \nonumber \\
\label{eq:S_DW2}
 \!\! &+& \!\! \quad \sum_{i=2}^{N-1} \bar \Psi_i \gamma_5
 \bigl[ (a_5 H_w P_- -1) \chi_i + (a_5 H_w P_+ +1) \chi_{i+1} \bigr] \\
 \!\! &+& \!\! \quad \bar \Psi_N \gamma_5 \bigl[ (a_5 H_w P_- -1) \chi_N
  + (a_5 H_w P_+ +1) (P_+ -m P_-) \chi_1 \bigr] \Bigr\} . \nonumber
\eea
In this case we introduce $\bar \chi_i$'s through $\bar \Psi_i =
\bar \chi_i (a_5 H_w P_- -1)^{-1} \gamma_5$ to arrive at
\bea
S_{DW} = \!\! &-& \!\! \biggl\{ \bar \chi_1 \bigl[ (P_- -m P_+) \chi_1 -
 (a_5 H_w P_- -1)^{-1} \gamma_5 \hat A \chi_1 - T^{-1} \chi_2 \bigr] \nonumber \\
 \!\! &+& \!\! \quad \sum_{i=2}^{N-1} \bar \chi_i \bigl[ \chi_i-
 T^{-1} \chi_{i+1} \bigr]
 + \bar \chi_N \bigl[ \chi_N - T^{-1} (P_+ -m P_-) \chi_1 \bigr] \biggr\} .
\label{eq:S_DW3}
\eea
Here, with the ordering being important,
\bea
T^{-1} = (1 - a_5 H_w P_-)^{-1} (1 + a_5 H_w P_+) = \begin{pmatrix}
 \tilde B + a_5^2 C \frac{1}{\tilde B} C^\dagger & a_5 C \frac{1}{\tilde B} \cr
 a_5 \frac{1}{\tilde B} C^\dagger & \frac{1}{\tilde B} \cr \end{pmatrix} ~,
\label{eq:Tinv_DW}
\eea
and thus
\bea
T = (1 + a_5 H_w P_+)^{-1} (1 - a_5 H_w P_-) = \begin{pmatrix}
 \frac{1}{\tilde B} & - a_5 \frac{1}{\tilde B} C \cr
 - a_5 C^\dagger \frac{1}{\tilde B} & a_5^2 C^\dagger \frac{1}{\tilde B} C
 + \tilde B \cr \end{pmatrix} ~.
\label{eq:T_DW}
\eea
where $\tilde B = 1 + a_5(B - M)$. $T$ is the usual domain wall fermion
transfer matrix in our conventions. The integration over the $\chi$ and
$\bar \chi$ fields proceeds as before, and we obtain
\bea
\frac{\det D^{(5)}_{DW}(m)}{\det D^{(5)}_{DW}(1)}
 = \det \left\{ \left[ D^{(4)}_{DW}(1) \right]^{-1} D^{(4)}_{DW}(m) \right\} .
\eea
with
\bea
&& \left[ D^{(4)}_{DW}(1) \right]^{-1} D^{(4)}_{DW}(m) = D_N(m) = \nonumber \\
\label{eq:D_N}
&& \frac{1}{2} \Biggl[1+m + (1-m) \gamma_5 \frac{T^{-N} -1}{T^{-N} + 1}
 + \gamma_5 \frac{2}{(T^{-N} + 1)(a_5 H_w P_- -1)} \gamma_5 \hat A \Biggr] = \\
&& \frac{1}{2} \Biggl[1+m + (1-m) \gamma_5 \tanh\left(-\frac{N}{2}
 \log T \right) + \gamma_5 \frac{2}{(T^{-N} + 1)(a_5 H_w P_- -1)}
 \gamma_5 \hat A \Biggr] = \nonumber \\
&& \frac{1}{2} \Biggl[1+m + (1-m) \gamma_5 \varepsilon_{N/2}(a_5 {\cal H})
 + \gamma_5 \frac{2}{(T^{-N} + 1)(a_5 H_w P_- -1)} \gamma_5 \hat A
 \Biggr] ~. \nonumber
\eea
In the last line, in analogy with (\ref{eq:T}), we have introduced an
${\cal H}$ by
\bea
T^{-1} = \frac{1 + a_5 {\cal H}}{1 - a_5 {\cal H}} ~.
\label{eq:T_calH}
\eea
Neglecting the term involving $\hat A(m)$ we recognize Neuberger's
4-d operator for domain wall fermions, written in two different ways.
{}From (\ref{eq:T_DW}) we find that ${\cal H}$ is given by
\bea
{\cal H} = \frac{1}{2 + a_5 H_w \gamma_5} H_w =
 H_w \frac{1}{2 + a_5 \gamma_5 H_w} ~.
\label{eq:calH}
\eea
The inversion, necessary for the computation of ${\cal H}$ should have
rather good condition number -- it corresponds to inverting a very
heavy Wilson fermion -- and thus should be rather quick.

We can again use $\hat A$ to project low-lying eigenvectors, this time
of ${\cal H}$. Let
\bea
{\cal H} v_i = \lambda_i v_i, \qquad T v_i = T_i v_i,
 \qquad \hat{P}_i = v_i v_i^\dagger ~.
\eea
with $T_i$ as in (\ref{eq:T_i}).
The projection can then be achieved by setting
\bea
\hat A(m) = (1-m) \gamma_5 \sum_i (a_5 H_w P_- -1) g_i \hat{P}_i ,
\eea
with
\bea
g_i &=& \frac{1}{2} \left[-\left(T_i^{-N} -1 \right) +
 \left(T_i^{-N} + 1\right)\epsilon(a_5 \lambda_i)\right] ~.
\label{eq:g_i}
\eea
$\hat A(m)$ vanishes again for $m=1$.

In complete analogy with the derivation from (\ref{eq:D_ov_inv}) to
(\ref{eq:D_ov5d_inv}), with the replacement $H_w \rightarrow H_w P_-$ we find
the relation between the inverse of $D_N$ and the 5-d domain wall Dirac
operator
\bea
D_N^{-1}(m) \!\! &=& \!\! \left[ D^{(4)}_{DW}(m) \right]^{-1}
 D^{(4)}_{DW}(1) \nonumber \\
 \!\! &=& \!\! \left\{ {\cal P}^{-1} \left[ D^{(5)}_{DW}(m) \right]^{-1}
 D^{(5)}_{DW}(1) {\cal P} \right\}_{11} ~.
\label{eq:D_N5d_inv}
\eea

Note that this is not the physical quark propagator $\< q \bar q \>$ as
obtained from domain wall fermions. Just like the overlap propagator, it
still needs a subtraction and multiplicative normalization. To make the
connection to $\< q \bar q \>$ explicite, we write the 1 for the
subtraction as
\bea
1 = \left\{ {\cal P}^{-1} \left[ D^{(5)}_{DW}(m) \right]^{-1}
 D^{(5)}_{DW}(m) {\cal P} \right\}_{11}
\eea
and thus
\bea
D_N^{-1}(m) - 1 = \left\{ {\cal P}^{-1} \left[ D^{(5)}_{DW}(m) \right]^{-1}
 \left[ D^{(5)}_{DW}(1) - D^{(5)}_{DW}(m) \right] {\cal P} \right\}_{11} .
\eea
{}From (\ref{eq:D_5d_DW}) we see that, neglecting the projection terms
$\hat A(m)$, that are not usually inlcuded for domain wall fermions,
\bea
\left[ D^{(5)}_{DW}(1) - D^{(5)}_{DW}(m) \right]_{ij} = (1-m) \left[
 P_- \delta_{iN} \delta_{j1} + P_+ \delta_{i1}\delta_{jN} \right] ~.
\eea
Therefore we obtain
\bea
D_N^{-1}(m) - 1 = (1-m) \left\{ {\cal P}^{-1} [ D^{(5)}_{DW}(m) ]^{-1}
 {\cal J} {\cal P} \right\}_{11} ~,
\eea
where ${\cal J}_{ij} = \delta_{i,N+1-j}$ is the inversion operator of the
fifth direction. Now from (\ref{eq:light}) we see that the physical
fermion degrees are given in terms of the domain wall boundary fermions
as
\bea
q = ({\cal P}^{-1} \Psi)_1 ~, \qquad\qquad
\bar q = (\bar \Psi {\cal J} {\cal P})_1 ~.
\eea
Hence we find
\bea
\< q \bar q \> = \left\{ {\cal P}^{-1} [ D^{(5)}_{DW}(m) ]^{-1} {\cal J}
 {\cal P} \right\}_{11} = \frac{1}{1-m} \left[ D_N^{-1}(m) - 1 \right] ~.
\eea

Finally, again in complete analogy to the steps used to relate $D_{ov}(m)$
to 5-d operators, we can do this for $D_{ov}(m)$, with the result
\bea
D_N(m) = \left\{ {\cal P}^{-1} \left[ D^{(5)}_{DW}(1) \right]^{-1}
 D^{(5)}_{DW}(m) {\cal P} \right\}_{11} ~.
\eea
With this, we can now compute, for example, the physically relevant
eigenvalues of domain wall fermions.

%\vskip 5mm
\newpage

\centerline{\bf \Large Eigenvalues of the pseudofermions}
\vskip 5mm

Here, we want to investigate a little more the properties of the
pseudofermions, needed to cancel the bulk contributions in the 5-d
domain wall fermion approches.

For both the standard domain wall fermion action (\ref{eq:S_DW}) and
for Borici's variant (\ref{eq:S_B}) we find for the pseudofermion
matrix, including the boundary conditions and recalling that the
projection $\hat A(1)$ vanishes,
\bea
D^{(5)}(1) D^{(5)\dagger}(1) = \begin{pmatrix}
X & Y & 0 & 0 &\cdots & 0 & 0 & -Y \cr
Y & X & Y & 0 &\cdots & 0 & 0 & 0 \cr
0 & Y & X & Y &\cdots & 0 & 0 & 0 \cr
\vdots & \vdots & \vdots & \vdots & \vdots & \vdots & \vdots & \vdots \cr
0 & 0 & 0 & 0 & \cdots & Y & X & Y \cr
-Y & 0 & 0 & 0 & \cdots & 0 & Y & X \cr
\end{pmatrix}
\label{eq:D_Ddag_5d}
\eea
with $X$ and $Y$ 4-d hermitian matrices,
\bea
Y &=& - P_+ D_+ - D_+^\dagger P_- = -\frac{1}{2} \left(a_5 D_w +
 a_5 D_w^\dagger + 2 \right) \nonumber \\
X &=& D_+ D_+^\dagger + 1 = a_5^2 D_w D_w^\dagger - 2 Y
\label{eq:X_Y_DW}
\eea
for the standard domain wall action, and
\bea
Y &=& D_+ P_+ D_-^\dagger + D_- P_- D_+^\dagger = a_5^2 D_w D_w^\dagger - 1
 \nonumber \\
X &=& D_+ D_+^\dagger + D_- D_-^\dagger = 2 a_5^2 D_w D_w^\dagger + 2 =
 4 a_5^2 D_w D_w^\dagger - 2 Y
\label{eq:X_Y_B}
\eea
for Borici's variant.

Now, let S be the shift (or translation) operator in the 5-th direction
with anti-periodic boundary condition:
\bea
S = \begin{pmatrix}
0 & 1 & 0 & \dots & 0 & 0 \cr
0 & 0 & 1 & \dots & 0 & 0 \cr
\vdots & \vdots & \vdots & \vdots & \vdots & \vdots \cr
0 & 0 & 0 & \dots & 0 & 1 \cr
-1 & 0 & 0 & \dots & 0 & 0 \cr \end{pmatrix}
\label{eq:shift}
\eea
It is easy to see that $[S, D^{(5)}(1) D^{(5)\dagger}(1) ] = 0$, and so
$S$ and $D^{(5)}(1) D^{(5)\dagger}(1)$ can be diagonalized simultaneously.
The eigenvalues of $S$ are $\exp\{i\pi(2k+1)/N\}$ for $k=0,\dots,N-1$,
because of the anti-periodic boundary conditions. The corresponding
eigenvectors are
\bea
\label{eq:ev_dwpf}
&& w(k)^T = \\
&& (v_4, v_4 \exp\{i\pi(2k+1)/N\}, v_4 \exp\{i\pi 2(2k+1)/N\}, \dots,
v_4 \exp\{i\pi(N-1)(2k+1)/N\} )^T \nonumber
\eea
with $v_4$ some 4-d vector. This also has to be an eigenvector of
$D^{(5)}(1) D^{(5)\dagger}(1)$. From (\ref{eq:D_Ddag_5d}), and using
(\ref{eq:X_Y_DW}) and (\ref{eq:X_Y_B}) we obtain the eigenvalue
equation for $v_4$
\bea
\left\{ J a_5^2 D_w D_w^\dagger - 2 Y \left[ 1 - \cos \left( \frac{\pi}{N}
(2k+1) \right) \right] \right\} v_4 = \lambda(k) v_4
\label{eq:4d_pf_ev}
\eea
Here $J=1$ for domain wall fermions, and $J=4$ for Borici's variant,
and we have indicated the dependence of $\lambda$ on $k$, the momentum
in the 5-th direction. Since the left-hand-side of (\ref{eq:4d_pf_ev})
remains unchanged under $k \rightarrow N-k-1$ we conclude that
$\lambda(N-k-1) = \lambda(k)$. Therefore, the eigenvalues of
$D^{(5)}(1) D^{(5)\dagger}(1)$ are all (at least) doubly degenerate.

Furthermore, for large $N$ and small $k$ the pseudofermion domain wall
eigenvalues closely track the Wilson eigenvalues, the eigenvalues of
$D_w D_w^\dagger$, since then $2 Y \left[ 1 - \cos \left( \frac{\pi}{N}
(2k+1) \right) \right] \approx Y \pi^2 (2k+1)^2/N^2$ in (\ref{eq:4d_pf_ev})
is only a small perturbation. In particular, at zero crossings of
the Wilson Dirac operator, a pair of pseudofermion domain wall eigenvalues
will go to zero for $N \to \infty$.

{}From the domain wall Dirac operator $D^{(5)}_{DW}(m)$ without the
projection matrix $\hat A(m)$, we can make a hermitian version,
$H^{(5)}_{DW}(m) = D^{(5)}_{DW}(m) {\cal J} \gamma_5$, with
${\cal J}_{ij} = \delta_{i,N+1-j}$ the inversion operator of the fifth
direction. From eq.~(\ref{eq:D_5d_DW}) it becomes
\bea
H^{(5)}_{DW}(m) = \begin{pmatrix}
m P_+ & 0 & 0 & 0 & \cdots & 0 & P_- & D_+ \gamma_5 \cr
0 & 0 & 0 & 0 & \cdots & P_- & D_+ \gamma_5 & -P_+ \cr
0 & 0 & 0 & 0 & \cdots & D_+ \gamma_5 & -P_+ & 0 \cr
\vdots & \vdots & \vdots & \vdots & \vdots & \vdots & \vdots & \vdots \cr
P_- & D_+ \gamma_5 & -P_+ & 0 & \cdots & 0 & 0 & 0 \cr
D_+ \gamma_5 & -P_+ & 0 & 0 & \cdots & 0 & 0 & -mP_- \cr
\end{pmatrix}
\label{eq:H_5d_DW}
\eea
Since $D^{(5)}_{DW}(m) D^{(5)\dagger}_{DW}(m) = [H^{(5)}_{DW}(m)]^2$, the
vectors $w(k)$ and $w(N-k-1)$ of eq.~(\ref{eq:ev_dwpf}) are degenerate
eigenvectors of $[H^{(5)}_{DW}(1)]^2$. To get the eigenvalues of
$H^{(5)}_{DW}(1)$ we need to compute the $2 \times 2$ matrix $\< w(k_1) |
H^{(5)}_{DW}(1) | w(k_2) \>$ with $k_1, k_2 = k, N-k-1$. We find
\bea
&& \< w(k_1) | H^{(5)}_{DW}(1) | w(k_2) \> = \nonumber \\
&& \sum_{j=1}^N
 {\rm e}^{ -\frac{i\pi}{N}(2k_1+1)(j-1)} {\rm e}^{ \frac{i\pi}{N}(2k_2+1)(N-j)}
 \< v_4 | P_- {\rm e}^{ -\frac{i\pi}{N}(2k_2+1)} + D_+ \gamma_5
 - P_+ {\rm e}^{ \frac{i\pi}{N}(2k_2+1)} | v_4 \> 
\eea
The sum over $j$ gives the constraint $k_2 = N-k_1-1$, and we thus find
the $2 \times 2$ matrix to be of the form
\bea
\begin{pmatrix} 0 & b \cr b^* & 0 \cr \end{pmatrix} \nonumber
\eea
and therefore the eigenvalues of the pseudofermion domain wall operator
come in $\pm$ pairs.

\end{document}
